\subsection*{サーバ管理総括}

%\writtenBy{\systemChief}{宇佐}{基史}
\writtenBy{\systemStaff}{宇佐}{基史}

2022年度春学期総会に掲げた以下の五つからなるサーバ管理方針に従い,
2022年度秋学期における活動を総括する.
\begin{itemize}
    \item サーバに用いるOSの選定
    \item 各種ミドルウェアの更新
    \item NASの運用
    \item 管理・運用の属人化を防ぐ
    \item 自主ドメインの更新など
\end{itemize}

\subsubsection*{サーバに用いるOSの選定}
調査の結果,
Cent Streamは保守性の点において本会の求める水準に達していないと結論付けられた.
代替案としてOracle Linux,Ubuntu Server,Ubuntuなどの
他のLinuxディストリビューションの可能性も模索している.
現状はCentOS 7による稼働である.
情報の煩雑さや局員間の非公式の場での調査や知見のやり取りが行われたなどの
点から一部見解や経緯が文書としてまとめられていない部分がある.
また,vpsの一部がCent OS8の試験運用で動いていたが,
停電対応二日目に停止させた.

\subsubsection*{各種ミドルウェアの更新}
脆弱性情報の確認,最新バージョン確認は停電対応二日目において,
十分に行われた.

\subsubsection*{NASの運用}
NASの外部接続に至った.

\subsubsection*{管理・運用の属人化を防ぐ}
サーバ勉強会資料やWiki,その他Googleドライブに残されている資料で一定の引き継ぎは行えている.
それら既存の資料を除き,
NASをはじめとした新サービスにおいては
Scrapboxなどの他のサービスに代替していく可能性を模索している.

\subsubsection*{自主ドメインの更新など}
更新業務は適切に行われた.
停電対応二日目に行われ,
引き継ぎも十分であることが確認できた.
