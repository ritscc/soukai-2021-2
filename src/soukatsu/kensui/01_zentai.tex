\subsection*{全体総括}

\writtenBy{\kensuiChief}{Park}{Jooinh}
%\writtenBy{\kensuiStaff}{Park}{Jooinh}

\begin{itemize}
  \item 平常活動の支援
  \item 会員が興味関心のある活動ができる環境づくり
  \item 発信力を養うための環境づくり
\end{itemize}

\subsubsection*{平常活動の支援}
平常活動の支援に関しては,プロジェクト活動の進捗管理やサポート,追い込み合宿,プロジェクト発表会の準備と進行を行った.

プロジェクト活動の進捗管理では,週報を用いてプロジェクト活動の進捗確認や問題の有無の確認を行い,
問題が確認された場合は,それを上回生会議の議題に上げることで問題の解決を図るという方針であったが,
週報の未提出が続き,上回生会議で議題に上げたが改善されない問題が起こった.
プロジェクトの活動自体は班員を通して問題なく行われていることが確認できていたため,催促だけは続ける形を取った.

2021年度春学期と同様,2021年度秋学期は,
プロジェクト活動,追い込み合宿,プロジェクト発表会はオンラインのみであり,
部屋取りは行わなかった.

追い込み合宿の進行は,開始や報告書のアナウンスを行った.

プロジェクト発表会では,進行と発表の録画を行わなかったが,次回からは行うことにする.
予め発表時間の目安を決定していたため,長時間や短時間の発表は無かった.
報告書は全ての班が提出できたが,準備が出来ず発表できないプロジェクト班があった.

\subsubsection*{会員が興味関心のある活動ができる環境づくり}
会員が興味関心のある活動ができる環境づくりに関しては,
春期勉強会を行う予定しており,まだ行っていない.

\subsubsection*{発信力を養うための環境づくり}
発信力を養うための環境づくりに関しては,LTとプロジェクト発表会の運営を行った.

LTは,毎週の定例会議中で行われた.

2021年度秋学期は,LTを担当週までに行うことができず,遅延して行われたものと最後までLTができないものが多かった.
中にはプロジェクトなどの他のクラブ活動は行っているにも関わらずLTを最後までできなった人員も多い.
Slackや定例会議を通して催促は続けて行ったが,逆に催促を負担に感じ,定例会議に参加しなくなった下回生もいたと考えられる.

また,例年通りLTは録画を撮りGoogleドライブに載せているが,2021年度秋学期からはNASに外部接続ができるようになったが,
混乱を防ぐためにGoogleドライブとNASの両方に載せるようにした.2022年度春学期からは完全にNASに移行する予定である.

また,LT意欲向上のため,LTアンケートを行い,入賞者には本会で購入する本の選択権を与えた.
2021年度秋学期は,LTアンケートの実施から結果の公表まで円滑に実施することが出来た.
これは,研究推進局が定例会議などのスケジュールを把握して事前に行動していたためである.

