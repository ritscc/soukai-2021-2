\subsection*{プロジェクト活動総括}

\writtenBy{\kensuiChief}{Park}{Jooinh}
%\writtenBy{\kensuiStaff}{Park}{Jooinh}


本項では本局におけるプロジェクト活動業務に関する2021年度春学期の総括を以下の点において述べる.

\begin{itemize}
  \item 企画書の募集
  \item 週報の回収・催促
  \item 会員のプロジェクト管理
  \item 発表の機会の提供
  \item 報告書の管理
\end{itemize}

\subsubsection*{企画書の募集}

2021年度春学期に設立した通年プロジェクトが2つで,2021年度秋学期に新しく5つの企画書が提出された.
企画書を局会議と上回生会議で確認を行い,全ての企画書に問題が無かったため,全てのプロジェクトを設立したが,
プロジェクト活動の参加募集の時点での人員不足により,2つの班が解散となった.

\subsubsection*{週報の回収・催促}

各プロジェクトリーダーは,プロジェクト活動の進捗確認や問題の有無の確認を行うために,
週報の提出が義務付けられている.
週報の回収にはGoogleフォームが用いられ,Slackのリマインダー機能を用いてリマインドが行われた.
週報の提出が遅延したプロジェクトもあったが,班員を通して正常に活動していることが確認できていたため,
上回性会議を通して毎週催促を行う程度に留めた.

\subsubsection*{会員のプロジェクト管理}

本局は,各会員がどのプロジェクトに所属しているかを把握し,
プロジェクトが途中で終了した場合などに所属していた会員のプロジェクト異動などを管理している.

2021年度秋学期に解散となったプロジェクト班の2つの内1つは2021年度春学期に設立された通年プロジェクトだったため,
上回性会議で解散の是非を話し合うべきだったが,本局でその手続きをせずに解散にさせてしまった.
今回はプロジェクト解散により所属するプロジェクトがなくなってしまった会員が居なかったため,
大きな問題にはならなかったが,次回からは注意するべき点である.

また,定例会議にだけ参加し,プロジェクト活動には参加しない会員や学期の途中などで新しく入ってくる会員がいた.
これらの定例会議,LT,プロジェクトの参加頻度が分からず,会員のプロジェクト管理に難航したため,
入会費の支払い記録以上に本会の活動に参加する会員が把握できるようなリストを次回から取り入れることが提案された.

\subsubsection*{発表の機会の提供}

プロジェクト活動の成果発表をプロジェクト発表会を通じて行った.
発表時間を予め制限していたため,発表は円滑に進行した.

発表ができないプロジェクト班もあったが,

\subsubsection*{報告書の管理}

プロジェクト活動の成果を記録として残すため,各プロジェクトリーダーには報告書の提出を義務としている.
全てのプロジェクト班が報告書を提出し,プロジェクト活動の成果発表の際に参加者全員でレビューし,
問題ないことを確認した.\textbf{}