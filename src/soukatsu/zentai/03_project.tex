\subsection*{プロジェクト活動総括}

\writtenBy{\president}{Park}{Jooinh}
%\writtenBy{\subPresident}{Park}{Jooinh}
%\writtenBy{\firstGrade}{Park}{Jooinh}
%\writtenBy{\secondGrade}{Park}{Jooinh}
%\writtenBy{\thirdGrade}{Park}{Jooinh}
%\writtenBy{\fourthGrade}{Park}{Jooinh}


\subsubsection*{全体総括}
2021年度秋学期のプロジェクト活動は,
2021年度春学期に設立された通年プロジェクトと2021年度秋学期に設立された半期プロジェクトの両方で行われた.
各プロジェクトには活動ごとに週報を提出することを義務付け,進捗確認を行った.
2021年度秋学期も2021年度春学期から引き続き新型コロナウイルス感染症流行のため,全ての活動をオンラインでの開催とした.

また,2021年度秋学期に設立された半期プロジェクトは活動開始が遅かったこともあり,
活動時期が短いため,成果を満足に達成できなかったと感じる班が存在した.

以下に2020年度秋学期に活動していたプロジェクトの一覧を示す.

\begin{itemize}
  \item DTM班
  \item Kaggle班
  \item LTプレゼン班
  \item TouchDesigner班
  \item サーバーサイドエンジニア養成班
\end{itemize}

プロジェクト活動の総括は以下の六つに分けて行う.

\begin{itemize}
  \item 目標の総括
  \item プロジェクトの内容
  \item 週報
  \item 報告書
  \item 追い込み合宿
  \item プロジェクト発表会
\end{itemize}

\subsubsection*{目標の総括}
202年度秋学期の目標は以下の三つであった.

\begin{itemize}
  \item 活動を通して技術力の向上を図る
  \item 個人のみならずグループ活動としての経験を得る
  \item 活動によって得られた成果を本会Webサイトを通して公開する
\end{itemize}

これらを踏まえた総括を以下に記す.

活動を通して技術力の向上を図るに関しては,
プロジェクト活動をオンラインのみでの活動に限定していたことから,
班員の活動成果に大きな差が存在する班が存在した.
しかし,程度に差はあれど,プロジェクト活動に参加した会員の技術力が向上したことから,
この目標は概ね達成できたと言える.

集団行動の重要性を学ぶに関しては,
プロジェクト活動の参加率は高く,達成できていたと言える.
しかし,途中からグループでの活動が少なく,個人活動がほとんどとなっていた班も存在した.
また,会員基準ではプロジェクト参加率は低くなかったものの,
プロジェクト活動の掛け持ちが多く,班員基準での参加率が低い班が存在した.

得られた成果を本会Webサイトを通して公開するに関しては,
プロジェクト活動報告書は提出されており,程なくWebサイトに公開する予定である.

\subsubsection*{プロジェクトの内容}
プロジェクトの内容については,春学期から引き続き全ての班において適切であった.

\subsubsection*{週報}
2021年度秋学期は週報を提出しない班が複数存在した.
研究推進局では続けて催促を行ったが,改善されなかった.
しかし,当該プロジェクトの班員とプロジェクトリーダーから活動が問題なく行われていることを確認していたため,
上回生会議で続けて催促を行うに留めた.

また,2019年度秋学期方針にて,週報が出ていないまたは週報に活動の継続が難しい旨が記述されていた場合,
プロジェクトのリーダーを上回生会議に招集し,
プロジェクトの存続を問うこととしており,
2021年度秋学期ではOS班が該当したが,
研究推進局で独自的に判断し,プロジェクトを解散させる事態が発生した.
OS班の解散によって所属プロジェクトがなくなった会員は存在しなかったため,
深刻な問題にはならなかった.

\subsubsection*{報告書}

報告書の提出をもってプロジェクト終了としており,
全ての班が報告書を提出したことにより,
これを達成できた.
報告書の内容に関しては,ほぼ問題はなかった.
報告書の形式については,PDF形式で提出するものとした.

\subsubsection*{追い込み合宿}
2月4日,5日にプロジェクト発表会に向けて班員に準備を行ってもらうために,オンラインで追い込み合宿を行った.

多くの会員が参加し,追い込み合宿で進捗を出していたことから追い込み合宿は十分な有用性があると言える.

\subsubsection*{プロジェクト発表会}

プロジェクト活動を通して得ることができた知見や技術を会内で共有する場として,オンラインでプロジェクト発表会を行った.

プロジェクト発表会は報告書のレビューを行う時間をとった後に発表,質疑応答を行うといった形式で行った.

レビューは5分間時間をとり,時間が足りなければ追加時間をとるといった形式で行った.

発表に関しては,全ての班が予め発表スライドを作成していたため,円滑に進行した.
また,発表時間を10分から20分に予め制限していたため,長時間や短時間の発表は無かった.

発表ができなかった班が一つ存在したが,報告書は提出されていたため,
発表は見送ることにした.
