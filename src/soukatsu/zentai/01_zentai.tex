\subsection*{2021年度秋学期活動総括}
 
\writtenBy{\president}{深田}{紘希}

本会の目的である「情報科学の研究,及びその成果の発表を活動の基本に会員相互の親睦を図り,
学術文化の創造と発展に寄与する」ことを達成するため,方針として以下の五つを立てた.
これらについてそれぞれ評価を行うことで2021年度秋学期の総括とする.

\begin{itemize}
    \item 親睦を深める
    \item 規律ある行動
    \item 自己発信力の向上
    \item 会員間の技術向上
    \item 外部への情報発信
\end{itemize}

\subsubsection*{親睦を深める}
    2021年度秋学期活動では,主にプロジェクト活動の実施や,クリスマス会,局活動の開催を通して会員間の親睦を図った.
    どのイベントも新型コロナウイルス感染症に最新の注意をはらい,Zoomでの実施や当時の大学の規定に沿って行われた.

    2021年度秋学期のプロジェクト活動は2021年度春学期活動に引き続きオンラインで実施された.
    対面での活動が行えないことから,班員同士で気軽にコミュニケーションを取ることができるような
    機会が少なかったため,親睦を深めるには至らなかった班が多かった.
    
    目的を達成した班は共通して毎週定期的にプロジェクト活動を実施していた.
    よってリモート環境下では班員の信頼を得るためにも,定期的に活動することが必要だと考えられる.

    クリスマス会はコミュニケーションを十分取ることができるように工夫された内容であった.
    しかし主な対象者である\firstGrade{}の参加がなかったことから効果は限定的なものであった.
    参加した会員は親睦を深める大変良い機会となった.
    のちのアンケートで,プレゼント交換会での出費が参加を辞退してしまうひとつの理由であることを確認できた.
    
    局活動は積極的に行われていた.対面で交流する機会もあったことから,親睦を深めることができた.

    一部の会員はZoomにおけるチャット機能や,会のDiscordサーバを活用して積極的にコミュニケーションを取っており,
    これによってより親睦を深めることができた.
    
    局活動や一部の会員間では目的を達成することができた一方,
    イベントの参加率が低く,例年に比べて目的を達成できたとはいえない.

\subsubsection*{規律ある行動}
    2021年度秋学期の方針として,遅刻・欠席連絡と備品整理,
    サークルルームの使用方法の三つの項目からなる行動規範を定めた.
    
    遅刻・欠席連絡は,理由が明記されていないものや,事後報告なども少なからず見られたものの,
    大半は開始時刻前に行われていた.
    参加率の低い会員は遅刻・欠席連絡も少なかった.
    
    2021年度秋学期からは,本格的にサークルルームを利用することが可能になった.
    サークルルームの使用方法については,ゴミの処理を怠っていた会員が一部発生していた.
    またサークルルーム予約のWebサービスがあるにもかかわらず,予約をせずに入室している会員も存在した.
    
    備品整理については,有志の会員により清掃が行われ整頓することができた.
    学生部によるサークルルーム点検においても,特に戒告を受けることはなかった.

\subsubsection*{自己発信力の向上}
    自己発信力を向上させるための機会として,2021年度秋学期活動では,
    ライトニングトーク(以下,LT)やプロジェクト発表会,Advent Calendar,会誌の制作を実施した.

    定例会議におけるLTは,参加率が高くなかったものの
    \firstGrade{}も意欲的に参加しており,自己発信力を向上させる良い機会となった.

    プロジェクト発表会では,定例会議と同程度の参加者が集まり,
    ひとつのプロジェクト班を除きプロジェクトの発表を滞りなく行うことができた.
    
    Advent Calendarは実施することができた一方,記事が不足したことから複数記事を書いた会員も存在した.

    会誌は担当者のリマインドもあり,参加率が高く内容も充実していた.
    2021年度は学園祭が開催され,会誌の配布を行うこととなった.
    用意していた部数を早々に配布することができ,本会Webサイト上での公開も行った.
    
    これらの活動によって,自己発信力を向上させることにつながったと考えられる.
    
\subsubsection*{会員間の技術向上}
    会全体の技術力を向上させることを目的として,LTやプロジェクト活動を実施した.

    定例会議におけるLTでは,
    担当者によるLTは少なかった一方で飛び入りLTは一定数存在し,どちらも内容が充実していた.

    プロジェクト活動は,全ての班がオンラインで十分な内容の活動を行い,
    プロジェクト発表会において,他会員に活動内容を共有することができた.

    また,技術力向上の場として学園祭ハッカソンを開催した.
    参加者は技術力向上につながったが,\firstGrade{}の参加が無かった.
    例年ハッカソンは\firstGrade{}の参加が少なめであるが,
    グループにメンターを付けたり事前勉強会を開催したりして,参加のハードルを下げる試みが必要である.
    
    全体としてリモート環境下でモチベーション維持が難しいなか,ある一定以上の技術向上がみられた.

\subsubsection*{外部への情報の発信}
    会外へ活動を発信する機会として,主に本会Webサイトと会公式Twitterが挙げられる.

    本会Webサイトと会公式Twitterでは,LTやイベントが行われる度にその様子が発信することが目標である.
    2021年度はこれらの頻度や内容が少なめであり,本会の活動を知ってもらうためにはさらなる発信が求められる.
     
