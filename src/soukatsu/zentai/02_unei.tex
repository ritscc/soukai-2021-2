\subsection*{運営総括}

%\writtenBy{\president}{斎藤}{竜也}
\writtenBy{\subPresident}{齋藤}{竜也}
%\writtenBy{\firstGrade}{斎藤}{竜也}
%\writtenBy{\secondGrade}{斎藤}{竜也}
%\writtenBy{\thirdGrade}{斎藤}{竜也}
%\writtenBy{\fourthGrade}{斎藤}{竜也}

2021年度秋学期の運営に関して以下の5点から総括を述べる.
\begin{itemize}
    \item 定例会議
    \item 上回生会議
    \item 局
    \item 企画
    \item 運営サポート
\end{itemize}

\subsubsection{定例会議}
春学期と変わらず,毎週木曜日に開催した.
対面での定例会議が困難なためZoomを用いて実施した.
内容に関しても特に変わることなく局・企画担当者からの連絡,LTであった.
参加人数は毎週会員の約6割ほどの人数であった.

イベントを含め各連絡の際に口頭のみでの告知・報告が多く見受けられた.
プレゼン資料などの視覚的な資料が望まれており,より分かりやすい告知を心がけるべきであったが,
各企画などにも大きく影響が出ることなく運営できていた.

全体を通して見ると,定例会議の運営は問題なく行えていたと考えている.

\subsubsection{上回生会議}
春学期同様,会内Discordを用いて毎週水曜日に開催した.
2021年度秋学期では各局長の出席に関しては問題なかった.
また,各企画担当者も呼び出しの際に問題なく出席できていた.

一部の執行部以外の\thirdGrade{}も参加しており,運営面から上回生会議は問題なく行えていたと言える.
しかしながら上回生会議内で開催する企画のKPTを失念していたためこの点のみ反省点としてあげられる.

\subsubsection{局}
2021年度秋学期ではこれまで問題とされていた局会議の頻度を下げ毎週開催の義務を科さずに必要性に応じて
局会議を開催することとした.
研究推進局に関してはプロジェクト活動の進捗確認などの関連でほぼ毎週開催されたが
それ以外の局については議題があり会議が必要な場合のみ局会議が開催された.
これについての問題は特に起こらなかった.

\subsubsection{企画}
2021年度秋学期では計画されていた企画に関して全てが開催された.
企画担当者は\secondGrade{}を中心に\thirdGrade{}がサポートする形で運営された.
懸念点として\firstGrade{}の参加がなかった企画がみられ引き継ぎの観点から十分に至っていない企画が存在することである.

\subsubsection{運営サポート}
2020年度に開催されなかった企画を中心に\thirdGrade{}はサポートに徹した.
ほぼ全ての企画において普段は\secondGrade{}が運営をし,問題が起きた時のみ\thirdGrade{}も対応する形で運営されていた.
\thirdGrade{}が介入しすぎずに運営できた.

\subsubsection{対面活動に関して}
2020年度から申請していた対面活動再開に関しての申請が2021年度夏期休暇頃に受理されたため,
実際にサークルルームの使用が可能になった.会員の体温管理とともに入室記録管理も行われた.
実際は同会員が繰り返し使用する結果となったが,使用条件からある程度は仕方ないと考えられる.