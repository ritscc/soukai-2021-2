\subsection*{本会規約の改正に関する議案}
\writtenBy{\kaikeiChief}{佐久間}{智也}

立命館コンピュータクラブ規約第二十四条に基づいて規約改正の発議を行う.

\subsubsection*{目的}
会員の減少により現在の運営体制を維持することが困難という状況が発生している.
そこで会計総務局を\kaikeiDepartment{}と\soumuDepartment{}に分離し,
\kaikeiDepartment{}を解体し,その業務を新たに執行部内に置く会計に移行することにより
局運営に必要な人員の削減を目的とする.,.

\subsubsection*{現状規約}
\subsection*{第八条(執行部)}
執行部は執行委員長一名,副執行委員長一名,会計総務局長一名,研究推進局長一名,渉外局長一名,システム管理局長一名を
以って構成され,本会の運営全般にたいして責任を負う.なお,執行委員長が必要と認めた場合には,副執行委員長が執行委員長の
職務を代行することとする.

\subsubsection*{第十条(会計総務局)}
会計局は会計総務局長一名,会計総務局員数名を以って構成され,本会の会計事務,本会の備品管理,対内活動全般,会議における書記を行う.

\subsubsection*{第十四条(総務局)}
総務局は総務局長一名,総務局員数名を以って構成され,本会の備品管理,対内活動全般,会議における書記を行う.

\subsection*{第三十八条(会計)}
本会の予算は会計総務局と上回生会議での議論を経て之を定め,決算は之を本会の会員の要望により公表するものとする.

\subsubsection*{新規約案}
\subsection*{第八条(執行部)}
執行部は執行委員長一名,副執行委員長一名,会計一名,総務局長一名,研究推進局長一名,渉外局長一名,システム管理局長一名を
以って構成され,本会の運営全般にたいして責任を負う.なお,執行委員長が必要と認めた場合には,副執行委員長が執行委員長の
職務を代行することとする.

\subsubsection*{第十条(総務局)}
総務局は総務局長一名,総務局員数名を以って構成され,本会の備品管理,対内活動全般,会議における書記を行う

\subsection*{第三十八条(会計)}
本会の予算は上回生会議での議論を経て之を定め,決算は之を本会の会員の要望により公表するものとする..

会計総務局が分離し,総務業務が総務局に移管されることにより規約の会計総務局の部分を総務局に名称変更する.
\begin{itemize}
\item 第九条(局体制)
\end{itemize}

会計総務局が分離し,会計業務が会計に移管されることより規約の会計総務局の部分を会計に名称変更する.
\begin{itemize}
\item 第三十六条(資産)
\item 第三十七条(会費・入会金)
\end{itemize}
