\subsection*{全体方針}

% \writtenBy{\kensuiChief}{Park}{Jooinh}
\writtenBy{\kensuiStaff}{Park}{Jooinh}

2022年度春学期の活動の中心になるものとして以下の三つを挙げる.
\begin{itemize}
  \item 平常活動の支援
  \item 会員が興味関心のある活動ができる環境づくり
  \item 発信力を養うための環境づくり
\end{itemize}

\subsubsection*{平常活動の支援}
平常活動の支援に関しては,週報を用いてプロジェクト活動の進捗確認や問題の有無の確認を行う.
問題が確認された場合は,それを上回生会議の議題に上げることで問題の解決を図る.
2021年度秋学期に引き続き,上回生会議での週報の確認も行う.

また,プロジェクトや定例会議などの出席率の低下への対策として,対面活動を推進するものとする.

\subsubsection*{会員が興味関心のある活動ができる環境づくり}
会員が興味関心のある活動ができる環境づくりに関しては,開催を希望する勉強会のアンケートをとり,
多くの意見が寄せられた分野の環境を充実させることで実現していく.
定例会議が対面で行われる場合は勉強会も対面で行うことを検討する.

\subsubsection*{発信力を養うための環境づくり}
発信力を養うための環境づくりに関しては,2021年度秋学期と同様にLTを実施する.
会員の意欲向上のために,LTアンケートを行う.
