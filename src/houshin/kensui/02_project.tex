\subsection*{プロジェクト活動方針}

%\writtenBy{\kensuiChief}{宮寺}{大樹}
\writtenBy{\kensuiChief}{Park}{Jooinh}

本項では本局におけるプロジェクト活動業務に関する2022年度春学期の方針を以下の点において述べる.

\begin{itemize}
  \item 企画書の募集
  \item 週報の回収・催促
  \item 会員のプロジェクト管理
  \item 発表の機会の提供
  \item 報告書の管理
\end{itemize}

\subsubsection*{企画書の募集}

企画書を募集するフォームを作成する.集まった企画書は局会議でレビューした後,上回生会議に回すこととする.
企画書の基準は全体方針のプロジェクト活動方針に従うものとする.
半期プロジェクトでは期限が短いため,基本的に通年のプロジェクトだけ募集し,半期プロジェクトは必要に応じてだけ募集する.

\subsubsection*{週報の回収・催促}

2021年度秋学期と同様,週報の回収・催促はSlackを通じて行う.ただし,進捗を会員全体へと共有するために局会議と上回生会議でも必ず行うこととする.

\subsubsection*{会員のプロジェクト管理}

プロジェクトの異動および,途中参加には本局と異動先のプロジェクトのリーダーの承認が必要である.
また,プロジェクトのリーダーの負担を減らすため,要望があれば本局員が活動を行う部屋の予約を代行する.
週報を通して,活動が芳しくないプロジェクトがあれば本局の方から上回生会議で報告し,上回生会議にて適切な処置を図る.
不適切な理由により活動をしていない班については,リーダーを上回生会議に呼び出し,存続の意思を問う.
無い場合,班員を呼び出した後に研究推進局の下,上回生会議にて適切な処理を行う.
ある場合は,班内での活動計画を再考後,上回生会議にて提出を行う.

\subsubsection*{発表の機会の提供}

2021年度秋学期に引き続きプロジェクト発表会を行う.
その際,事前に配布された報告書をその場で読む時間を設ける,スライド発表,質疑応答との形式で執り行う.
ただし,対面で行うかオンラインで行うかは状況を観察しながら決定する.
報告書のテンプレートを各プロジェクトリーダーに配布,PDFでレビューした後,対面で行うのであれば2部印刷する.なお,テンプレートの配布の時期は定例会議にて連絡を行う.


\subsubsection*{報告書の管理}

フォントサイズ,マージン,行間などの報告書のテンプレートを研究推進局が用意し,それに準じた報告書を作成してもらう.
作成された報告書はプロジェクト発表会中にレビュー,修正し,修正期間を経た後にPDFで提出してもらう.
報告書を提出するまでの執筆形式は問わない.
集めたPDFは渉外局に依頼し,本会Webサイトに公開する.
報告書では載せられない制作物は,上回生会議にて著作権などの確認を行い,Webサイトでの公開を推奨する.
