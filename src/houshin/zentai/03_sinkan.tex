\subsection*{新歓方針}

%\writtenBy{\president}{山本}{京介}
%\writtenBy{\subPresident}{山本}{京介}
%\writtenBy{\firstGrade}{山本}{京介}
\writtenBy{\secondGrade}{山本}{京介}
%\writtenBy{\thirdGrade}{山本}{京介}
%\writtenBy{\fourthGrade}{山本}{京介}

\subsubsection*{目的}
新入生歓迎活動を行う目的として以下の二つを挙げる.
\begin{itemize}
\item 新入生に会の活動内容について知ってもらう
\item 新入生に会に興味を持ってもらう
\end{itemize}
これは本会の活動を会外に示すとともに,
新入生に会の方針を知ってもらった上で歓迎するために重要なものである.
加えて,人材確保は会の活動を維持していく上でも重要なものである.

\subsubsection*{目標}
目標に関しては,以下の四つを挙げる
\begin{itemize}
\item 企画に参加してもらう
\item 気軽にサークルルームに来てもらう
\item 本会でやりたいことを見つけてもらう
\item 新入生の中長期的な定着
\end{itemize}
目標達成のために,サークルルームを見学できることを新入生にアピールする.
サークルルームの場所がわかりづらいため,SNSやホームページで地図や動画を使って,詳しく説明する.
サークルルームではクライアントPCやサーバ,本棚の紹介をする.
アポイントメントなしで新入生がこれるよう,常に\secondGrade{}の会員が少なくとも二人程度いるようにする.

\subsubsection*{手法}
大学側が主催するイベントに積極的に参加する.
そのために定期的にメールを確認し,申請期限を守る.
参加したイベントでは会誌を配布や,成果物の発表などを行う.
イベントの開催はSNSやホームページで告知する.
また,ブースやサークルルームの入り口を華やかにし,どういったサークルかわかりやすくする.
サークルルームに見学に来た新入生のために,サークルルームにいる上回生は名札をぶら下げ,上回生だとわかりやすくする.
サークルルームでの新入生対応では,本会に関することだけでなく,大学生活全体に関する話もする.
