\subsection*{Welcomeゼミ方針}


\writtenBy{\firstGrade}{古川}{聡悟}

Welcomeゼミを行う目的は,例年通り以下の2点である.

\begin{itemize}
    \item 新入生に本会に興味を持ってもらうこと
    \item 新入生に本会の活動を体験してもらうこと
\end{itemize}

\subsubsection*{目標}
\begin{itemize}
    \item 新入生に居心地の良い空間にすること
    \item 新入生に活動に積極的に参加してもらうこと
    \item 新入生の中長期的な定着
\end{itemize}

2021年度はコロナ禍の影響を受け,オンラインでの活動が中心的であった.
2022年度も同様の状況が続く可能性があるため,Discordなどの使用を奨励していく.
一方で,関係密接化のため対面での交流も増やしていく.

\subsubsection*{手段}
Welcomeゼミの具体的な手段について,方式と内容を順に述べる.

まず,方式については例年通り,新入生の希望分野に適した上回生をあてがい開発を行う.
2019年度以前は原則\secondGrade{}を割り振ることとなっていたが,2022年度は2020と2021年度を踏襲し割り当てる会員を\secondGrade{}に限定しない.
1人の上回生に負担が集中しないように考慮する.そのため,新入生の進捗状況は上回生全体で共有,管理する.
教導の効率化のためにも,新入生に分かりやすく教えることを念頭に置く.
エナジードリンクの類については推奨せず,エディタやプログラミング言語に関する複雑な話題は避ける.
対面での作業も可能とする.

次に,内容についても2021年度同様新入生の希望を調査し,該当する分野を得意とする上回生をあてがう.
但し,得意な分野の先輩がいない場合は除く.
その後は新入生のペースに合わせ開発を進める.
友好的に接しつつ難易度は全体的に易化し,プロダクト制作に固執せず一定の成果を上げることを目的とする.
最終日には成果発表を行う.コロナ禍の現況を鑑み,オンラインでの開催が予定されるが,状況が好転した際はその限りではない.
原則全員が行い,成果の共有と全体での発表を経験してもらう.
2021年度も原則全員であり,2022度も新入生の経験を積むためにも原則全員が発表することが望ましい.

また,連絡手段は初回は学内メールを使用するが,以降はSlackやDiscord,Zoomなどのツールも臨機応変に対応する.
