\subsection*{立命の家方針}

%\writtenBy{\president}{宮寺}{大樹}
%\writtenBy{\subPresident}{宮寺}{大樹}
%\writtenBy{\firstGrade}{宮寺}{大樹}
\writtenBy{\secondGrade}{山本}{京介}
%\writtenBy{\thirdGrade}{宮寺}{大樹}
%\writtenBy{\fourthGrade}{宮寺}{大樹}

\subsubsection*{概要}
立命の家は毎年立命館大学で開催される小学生を対象とした企画である.
2021年度は新型コロナウイルス感染症拡大防止のため,オンラインで開催された.
立命館大学に所属する学術団体やプロジェクト団体は
小学生に科学や英語,ものづくりの楽しさを知ってもらうために
様々な企画を行い,小学生と交流を深める.

\subsubsection*{目的と目標}
立命の家企画に参加する目的は,本会が学術部公認団体として
求められる還元活動の義務を果たすことである.
この目的の達成のために,本企画に参加する小学生に
本会の活動と情報技術に興味関心を向けてもらうことを目標とする.
加えて,小学生に教える体験を通して会員の教える能力の向上をはかり,
今後の還元活動を円滑にできるようになることを目標とする.

\subsubsection*{実施内容と提案}
本会ではプログラミング体験を実施する.
2021年度の立命の家において,プログラミングを小学校低学年に教えるのが困難であったことを受け,
対象は小学4年生からとし,15人を受け入れる.
教材は例年通りScratchを用いる.
2021年度と同様に,数グループに分け,一つの課題をグループごとに進行していく形にする.

\subsubsection*{改善点}
2021年度の反省点を踏まえ,
\begin{itemize}
  \item 資料の準備
  \item 人員の確保
\end{itemize}
を改善点とする.
2021年度は資料の準備が直前になってしまったため,リハーサルの時間があまりとれなかった.
そこで,1,2週間程の余裕をもって資料準備を完了し,リハーサルをする時間を確保する.
また,人員が多いほど,アクシデントに対応しやすいため,できるだけ多く人員を確保する.

\subsubsection*{役割}
立命の家で会員に割り当てられる役職は以下の通りである.
\begin{itemize}
  \item 担当者
  \item 企画リーダー
  \item 企画スタッフ
\end{itemize} 

立命の家担当者は新\secondGrade{}に割り当てられ,立命の家実行委員として参加する.週1回の実行委員会に出席し,立命の家全体の企画と運営をする.
当日は企画には参加せず,サポートに専念する.

企画リーダーは新\secondGrade{}に割り当てられる.企画の立案及び主導を行い,当日は企画の進行をする.

企画スタッフは新\firstGrade{}以上の全ての会員に割り当てられ,企画リーダーのサポートをする.
当日は小学生たちに企画の説明をし,小学生からの質問があれば対応する.
