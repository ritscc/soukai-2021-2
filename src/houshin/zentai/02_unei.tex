\subsection*{運営方針}

%\writtenBy{\president}{山本}{京介}
%\writtenBy{\subPresident}{山本}{京介}
%\writtenBy{\firstGrade}{山本}{京介}
\writtenBy{\secondGrade}{山本}{京介}
%\writtenBy{\thirdGrade}{山本}{京介}
%\writtenBy{\fourthGrade}{山本}{京介}

2022年度の運営について,以下の4点から方針を述べる.
\begin{itemize}
    \item 定例会議
    \item 上回生会議
    \item 局
    \item 企画
\end{itemize}

\subsubsection*{定例会議}
2022年度においても,2021年度同様週1回の定例会議を行う.
定例会議では,局や企画からの連絡や会員全体ですべき議決,LTなどを行う.
木曜の18時から行うが,参加率によっては変更を行う.
定例会議で的確な情報を共有するため,各局で確実な情報共有を行う.

\subsubsection*{上回生会議}
2022年度においても,2021年度同様週1回の頻度にて上回生会議を行う.
執行部及び企画担当者は全員参加とする.
欠席の場合は必ず代理人を立てるようにする.
議決権のない会員に関しても2021年度同様参加の意志があればその出席を認めることとする.
局長は局会議内で上回生会議での議題を共有し,局員も議題内容を把握できるよう努める.
また,企画書の提出は会議前日までと定め,企画書のレビュー時間の短縮と円滑な議事進行に努める.

\subsubsection*{局}
2022年度は安定した局員確保のため,夏期休暇前に配属を行う.
2021年度は会員が安定しないうちに配属を行ったため,
会員の離脱が発生し,新入生がいない局が発生してしまった.
そこで,ある程度遅めであるが,無理のない引き継ぎが行える夏期休暇前に配属を行うこととする.
希望調査は6月に行い,執行部で面談を行ったあと,上回生会議で配属先を決定する.
配属の際には,定例会議の参加状況を参考にする.

局会議は毎週の開催を強制しないが,議題があれば行うようにする.
あまりにも長い期間開催が無いようであれば,上回生会議で催促を行う.

\subsubsection*{企画}
本会外部との関わりがある行事に関しては\secondGrade{}の中で担当者を定め,その企画を行う.
2021年度の同企画担当者がそのサポートを行う.
企画書提出から企画の進捗は随時上回生会議にて報告する.また企画終了後にはKPTを上回生会議にて担当者を交えて行う.
