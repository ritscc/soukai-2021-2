\subsection*{全体方針}
\subsection*{全体方針}

総括とは?
前学期に立てた方針(青字)通りに活動ができたかを振り返る
総会は全体,局,回生ごとに総括と方針立てを行う
参考:2020年度秋学期活動総括

\#全体総括(きょうすけ)
\# 全体方針 (きょうすけ)

\#\# 会内総括
\#\# 会内方針
秋学期は対面活動が再開される可能性があるため,対面活動に対応できるように準備をする
局として対面で活動し,交流できる機会を増やした
局コンパ
勉強会

\#\# 局内総括
\#\# 局内方針
サーバ管理の徹底
局外にリソースを貸し出すことができるように整備する
その他会員からの要望に可能な限り対応する
本会DCの運用を開始する
NASの外部接続を始めた
過去のLTがリモート環境で見れるようになった


クライアントPCのメンテナンス
対面活動が再開されたら定期的に行う
サークルルームに適宜入室し,Windows Update を行った

\# 局会議総括(lufe)
\# 局会議方針 (lufe)
積極的な参加
秋学期も同様に無断欠席がないようにする
積極的な参加
時折出席が芳しくないため,無断欠席を無くし,時間を決定する

毎週開催
毎週開催ができたといっても過言ではない
後半の方は毎週ではなかったが,議題量が少なかったため問題はなかった

\# 勉強会総括(yumin)
\# 勉強会方針 (yumin)
サーバ構築勉強会を行う
可能であれば対面で行う
\firstGrade{}は対面で参加でき,実機に触れながら勉強会を進められた

\# サーバ管理総括
\# サーバ管理総括 (USA)
Cent OS以外のOSに切り替えたほうがいいかを調査する
Cent OSのままなら7で続行する
春学期の調査から進展はない
各種ミドルウェアの更新
停電対応時に行った

DCの運用
NASの外部接続を開始した

管理・運用の属人化を防ぐ
引き継ぎ資料を完成させる
サーバ勉強会資料やWiki,その他Googleドライブに残されている資料である程度の引き継ぎができている
一部の資料に関しては内容が更新されていない
他のサービスに変わり始めている可能性がある(Scrapboxとか)
既存の資料は新サービスに対応できてない

停電対応二日目
vpsの一部がCent OS8の試験運用で動いていたが,停止させた

\# 引き継ぎ総括(TZ)
\# 引き継ぎ方針 (TZ)
作成された引き継ぎ資料を活用し,またその更新を怠らないようにする
停電対応,ドメインの対応は引き継ぎ資料をもとに行うことができた
会内サービス(メーリス,本会アカウント,Googleドライブ)に関しては担当者感での引き継ぎをもとに行うことができた
資料作りたい(方針で詳細を記述する)

(てんちょ先輩より)GitHubを中心に文書を残していく.公開が不適切なものをWikiにするようにして差別化していくべきであろう.
GitHubの代わりにScrapboxに文書を残した

\# 新局員総括(TZ)
\# 新局員方針 (TZ)
局会議を通して,基礎知識の共有を行う.
システム管理局に必要な知識は局会議で共有しつつ,発展的な知識の共有は勉強会で行った.

実際に運用に携わることで知識の理解を深める.
勉強会である程度運用に携わる機会を設けた

%\writtenBy{\systemChief}{羽田}{秀平}
%\writtenBy{\systemStaff}{羽田}{秀平}

%\writtenBy{\systemChief}{羽田}{秀平}
%\writtenBy{\systemStaff}{羽田}{秀平}
